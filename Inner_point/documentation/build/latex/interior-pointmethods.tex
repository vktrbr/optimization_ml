%% Generated by Sphinx.
\def\sphinxdocclass{report}
\documentclass[letterpaper,10pt,english,openany,oneside]{sphinxmanual}
\ifdefined\pdfpxdimen
   \let\sphinxpxdimen\pdfpxdimen\else\newdimen\sphinxpxdimen
\fi \sphinxpxdimen=.75bp\relax
\ifdefined\pdfimageresolution
    \pdfimageresolution= \numexpr \dimexpr1in\relax/\sphinxpxdimen\relax
\fi
%% let collapsible pdf bookmarks panel have high depth per default
\PassOptionsToPackage{bookmarksdepth=5}{hyperref}

\PassOptionsToPackage{warn}{textcomp}
\usepackage[utf8]{inputenc}
\ifdefined\DeclareUnicodeCharacter
% support both utf8 and utf8x syntaxes
  \ifdefined\DeclareUnicodeCharacterAsOptional
    \def\sphinxDUC#1{\DeclareUnicodeCharacter{"#1}}
  \else
    \let\sphinxDUC\DeclareUnicodeCharacter
  \fi
  \sphinxDUC{00A0}{\nobreakspace}
  \sphinxDUC{2500}{\sphinxunichar{2500}}
  \sphinxDUC{2502}{\sphinxunichar{2502}}
  \sphinxDUC{2514}{\sphinxunichar{2514}}
  \sphinxDUC{251C}{\sphinxunichar{251C}}
  \sphinxDUC{2572}{\textbackslash}
\fi
\usepackage{cmap}
\usepackage[T1]{fontenc}
\usepackage{amsmath,amssymb,amstext}
\usepackage{babel}



\usepackage{tgtermes}
\usepackage{tgheros}
\renewcommand{\ttdefault}{txtt}



\usepackage[Bjarne]{fncychap}
\usepackage{sphinx}

\fvset{fontsize=auto}
\usepackage{geometry}
\usepackage{tikz}\usetikzlibrary{shapes,positioning}\usepackage{amsmath}

% Include hyperref last.
\usepackage{hyperref}
% Fix anchor placement for figures with captions.
\usepackage{hypcap}% it must be loaded after hyperref.
% Set up styles of URL: it should be placed after hyperref.
\urlstyle{same}

\addto\captionsenglish{\renewcommand{\contentsname}{Contents:}}

\usepackage{sphinxmessages}
\setcounter{tocdepth}{3}
\setcounter{secnumdepth}{3}


\title{Interior\sphinxhyphen{}point methods}
\date{Apr 26, 2022}
\release{1.0.0}
\author{Victor Barbarich, Adelina Tsoi}
\newcommand{\sphinxlogo}{\vbox{}}
\renewcommand{\releasename}{Release}
\makeindex
\begin{document}

\pagestyle{empty}
\sphinxmaketitle
\pagestyle{plain}
\sphinxtableofcontents
\pagestyle{normal}
\phantomsection\label{\detokenize{index::doc}}



\chapter{inner\_point}
\label{\detokenize{links:inner-point}}\label{\detokenize{links::doc}}

\chapter{algorithms module}
\label{\detokenize{links:module-Inner_point.algorithms}}\label{\detokenize{links:algorithms-module}}\index{module@\spxentry{module}!Inner\_point.algorithms@\spxentry{Inner\_point.algorithms}}\index{Inner\_point.algorithms@\spxentry{Inner\_point.algorithms}!module@\spxentry{module}}\index{gradient() (in module Inner\_point.algorithms)@\spxentry{gradient()}\spxextra{in module Inner\_point.algorithms}}

\begin{fulllineitems}
\phantomsection\label{\detokenize{links:Inner_point.algorithms.gradient}}\pysiglinewithargsret{\sphinxbfcode{\sphinxupquote{gradient}}}{\emph{\DUrole{n}{f}}, \emph{\DUrole{n}{x}}, \emph{\DUrole{n}{delta\_x}\DUrole{o}{=}\DUrole{default_value}{1e\sphinxhyphen{}08}}}{}
\sphinxAtStartPar
Returns the gradient of the function at a specific point x

\sphinxAtStartPar
A two\sphinxhyphen{}point finite difference formula that approximates the derivative
\begin{equation}\label{equation:links:links:0}
\begin{split}\displaystyle \frac{\partial f}{\partial x} \approx {\frac {f(x+h)-f(x-h)}{2h}}\end{split}
\end{equation}
\sphinxAtStartPar
Gradient
\begin{equation}\label{equation:links:links:1}
\begin{split}\displaystyle \nabla f = \left[\frac{\partial f}{\partial x_1} \enspace \frac{\partial f}{\partial x_2}
\enspace \dots \enspace \frac{\partial f}{\partial x_n}\right]^\top\end{split}
\end{equation}\begin{quote}\begin{description}
\item[{Parameters}] \leavevmode\begin{itemize}
\item {} 
\sphinxAtStartPar
\sphinxstyleliteralstrong{\sphinxupquote{f}} (\sphinxstyleliteralemphasis{\sphinxupquote{Callable}}\sphinxstyleliteralemphasis{\sphinxupquote{{[}}}\sphinxstyleliteralemphasis{\sphinxupquote{{[}}}\sphinxstyleliteralemphasis{\sphinxupquote{numpy.ndarray}}\sphinxstyleliteralemphasis{\sphinxupquote{{]}}}\sphinxstyleliteralemphasis{\sphinxupquote{, }}\sphinxstyleliteralemphasis{\sphinxupquote{numbers.Real}}\sphinxstyleliteralemphasis{\sphinxupquote{{]}}}) \textendash{} function which depends on n variables from x

\item {} 
\sphinxAtStartPar
\sphinxstyleliteralstrong{\sphinxupquote{x}} (\sphinxstyleliteralemphasis{\sphinxupquote{numpy.ndarray}}) \textendash{} n \sphinxhyphen{} dimensional array

\item {} 
\sphinxAtStartPar
\sphinxstyleliteralstrong{\sphinxupquote{delta\_x}} (\sphinxstyleliteralemphasis{\sphinxupquote{numbers.Real}}) \textendash{} precision of two\sphinxhyphen{}point formula above (delta\_x = h)

\end{itemize}

\item[{Returns}] \leavevmode
\sphinxAtStartPar


\item[{Return type}] \leavevmode
\sphinxAtStartPar
numpy.ndarray

\end{description}\end{quote}

\end{fulllineitems}

\index{jacobian() (in module Inner\_point.algorithms)@\spxentry{jacobian()}\spxextra{in module Inner\_point.algorithms}}

\begin{fulllineitems}
\phantomsection\label{\detokenize{links:Inner_point.algorithms.jacobian}}\pysiglinewithargsret{\sphinxbfcode{\sphinxupquote{jacobian}}}{\emph{\DUrole{n}{f\_vector}}, \emph{\DUrole{n}{x}}, \emph{\DUrole{n}{delta\_x}\DUrole{o}{=}\DUrole{default_value}{1e\sphinxhyphen{}08}}}{}
\sphinxAtStartPar
Returns the Jacobian matrix of a sequence of m functions from f\_vector by n variables from x.
\begin{equation}\label{equation:links:links:2}
\begin{split}\displaystyle \nabla f = \left[\frac{\partial f}{\partial x_1} \enspace \frac{\partial f}{\partial x_2}
\enspace \dots \enspace \frac{\partial f}{\partial x_n}\right]^\top\end{split}
\end{equation}\begin{quote}\begin{description}
\item[{Parameters}] \leavevmode\begin{itemize}
\item {} 
\sphinxAtStartPar
\sphinxstyleliteralstrong{\sphinxupquote{f\_vector}} (\sphinxstyleliteralemphasis{\sphinxupquote{Sequence}}\sphinxstyleliteralemphasis{\sphinxupquote{{[}}}\sphinxstyleliteralemphasis{\sphinxupquote{Callable}}\sphinxstyleliteralemphasis{\sphinxupquote{{[}}}\sphinxstyleliteralemphasis{\sphinxupquote{{[}}}\sphinxstyleliteralemphasis{\sphinxupquote{numpy.ndarray}}\sphinxstyleliteralemphasis{\sphinxupquote{{]}}}\sphinxstyleliteralemphasis{\sphinxupquote{, }}\sphinxstyleliteralemphasis{\sphinxupquote{numbers.Real}}\sphinxstyleliteralemphasis{\sphinxupquote{{]}}}\sphinxstyleliteralemphasis{\sphinxupquote{{]}}}) \textendash{} a flat sequence, list or tuple or other containing m functions

\item {} 
\sphinxAtStartPar
\sphinxstyleliteralstrong{\sphinxupquote{x}} (\sphinxstyleliteralemphasis{\sphinxupquote{numpy.ndarray}}) \textendash{} an n\sphinxhyphen{}dimensional array. The specific point at which we will calculate the Jacobian

\item {} 
\sphinxAtStartPar
\sphinxstyleliteralstrong{\sphinxupquote{delta\_x}} (\sphinxstyleliteralemphasis{\sphinxupquote{numbers.Real}}) \textendash{} precision of gradient

\end{itemize}

\item[{Returns}] \leavevmode
\sphinxAtStartPar
the Jacobian matrix according to the above formula. Matrix n x m

\item[{Return type}] \leavevmode
\sphinxAtStartPar
numpy.ndarray

\end{description}\end{quote}

\end{fulllineitems}



\renewcommand{\indexname}{Python Module Index}
\begin{sphinxtheindex}
\let\bigletter\sphinxstyleindexlettergroup
\bigletter{i}
\item\relax\sphinxstyleindexentry{Inner\_point.algorithms}\sphinxstyleindexpageref{links:\detokenize{module-Inner_point.algorithms}}
\end{sphinxtheindex}

\renewcommand{\indexname}{Index}
\printindex
\end{document}